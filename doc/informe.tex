% Created 2022-03-22 mar 14:57
% Intended LaTeX compiler: pdflatex
\documentclass[conference]{IEEEtran}
\usepackage[utf8]{inputenc}
\usepackage[T1]{fontenc}
\usepackage{graphicx}
\usepackage{longtable}
\usepackage{wrapfig}
\usepackage{rotating}
\usepackage[normalem]{ulem}
\usepackage{amsmath}
\usepackage{amssymb}
\usepackage{capt-of}
\usepackage{hyperref}
\input{~/org/latex/author_TeoCir2_Riedinger.tex}
\input{~/org/latex/ieee.tex}
\date{\today}
\title{Ejercicios Ruido}
\hypersetup{
 pdfauthor={},
 pdftitle={Ejercicios Ruido},
 pdfkeywords={},
 pdfsubject={},
 pdfcreator={Emacs 27.2 (Org mode 9.6)}, 
 pdflang={Spanish}}
\begin{document}

\maketitle
\tableofcontents


\section{Ejercicio 1}
\label{sec:org8b4a3b6}

Calcule la ganancia en \(dB\) de un amplificador ideal que opera a la frecuencia de 150MHz cuando en su entrada se inyecta una potencia de 1mW obteniendo a la salida una potencia de 1W. Calcule la potencia que debemos inyectar en la entrada para obtener 4W sobre la salida.

La ganancia en \(dB\) se puede calcular a partir de los valores de potencia de entrada y salida, independientemente de la frecuencia de operación:

\begin{equation}
    G[dB] = 10 \: \log{\left( \frac{P_o}{P_i} \right)} =
          = 10 \: \log{\left( \frac{1}{1\times 10^{-3}} \right)}
          = 30 \: [dB]
\end{equation}

Ahora, manteniendo la ganancia \(G = 30 \: [dB]\), la potencia de entrada \(P_i\) para que la salida sea \(P_o = 4 \: [W]\) será dependiente del valor (inicial) de ganancia en veces. Este será:

\begin{equation}
    G[veces] = \frac{P_o}{P_i} = \frac{1}{1 \times 10^{-3}} = 1000
\end{equation}

De esta forma:

\begin{equation}
    P_i = \frac{P_o'}{G[veces]} = \frac{4}{1000} = 4 \: [mW]
\end{equation}
\section{Ejercicio 2}
\label{sec:orgcb6dd78}
Calcule la tensión \(V_p\), \(V_{pp}\) (pico a pico), \(V_{rms}\), su equivalente en \(dBm\) y \(dBW\) de una señal senoidal de \(2\:[mW]\) (miliwatt) sobre una línea de transmisión de 50 ohms cuya terminación se encuentra adaptada.

A partir del valor de potencia y resistencia dado:

\begin{equation}
    P = \frac{V_{rms}^2}{R} \rightarrow
    V_{rms} = \sqrt{P \: R} = \sqrt{2\times 50} =
\end{equation}
\end{document}
